\begin{abstract}
    The development of genomic selection (GS) methods has allowed
 plant breeding programs to select favorable lines using genomic data before performing field trials. Improvements in genotyping technology have yielded high-dimensional genomic marker data which can be difficult to incorporate into statistical models. In this paper, we investigated the utility of applying dimensionality  reduction (DR) methods as a pre-processing step for GS methods. We compared five DR methods and studied the trend in the prediction accuracies of each  method as a function of the number of features retained. The effect of DR methods was studied using three models that involved the main effects of line, environment, marker, and the genotype by environment interactions. The methods were applied on a real data set containing 315 lines phenotyped in nine environments with 26,817 markers each. Regardless of the DR method and prediction model used, only a fraction of features was sufficient to achieve maximum correlation. Our results
 underline the usefulness of DR methods as a key pre-processing step in GS models to improve computational efficiency in the face of ever-increasing size of genomic data.  
 
\end{abstract}
